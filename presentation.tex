\documentclass[compress,notes=hide]{beamer}

\usepackage{amsmath}
\usepackage{amssymb}
\usepackage{amsfonts}
\usepackage{amsthm}
\usepackage{enumerate}
\usepackage[T1]{fontenc}
\usepackage[utf8]{inputenc}
\usepackage{verbatim}
\usepackage[retainorgcmds]{IEEEtrantools}
\usepackage{wasysym}


%% Layout setting
%\definecolor{ETHblue}{RGB}{51,89,148}           % #335994
%\definecolor{ETHlightblue}{RGB}{114,129,192}    % for Department-Logo
%\definecolor{ETHtitleblue}{RGB}{0,49,91}        % for title (#00315b)
%\definecolor{ETHbrown}{RGB}{148,116,51}
%\definecolor{ETHred}{RGB}{161,82,71}
%\definecolor{ETHdarkgray}{rgb}{0.282,0.322,0.361}    % for box (#48525c)
%--------------------
\usetheme{Antibes}
%\useinnertheme{rectangles}
\usecolortheme{beaver}
%\setbeamercolor{alerted text}{fg=red!75!green}
\usefonttheme{professionalfonts}
%\setbeamertemplate{footline}{\insertframenumber/\inserttotalframenumber}
\beamertemplatenavigationsymbolsempty

\setbeamertemplate{bibliography item}{}

\usepackage{amsmath}
\usepackage{amssymb}
\usepackage{amsfonts}
\usepackage{dsfont}


%------------ New Commands for the math environments
\newcommand{\ergo}{\rightarrow \quad}
\newcommand{\tergo}{$\quad \rightarrow \quad$}
\newcommand{\C}{\mathbb{C}}
\newcommand{\R}{\mathbb{R}}
\newcommand{\Z}{\mathbb{Z}}
\newcommand{\N}{\mathbb{N}}
\newcommand{\K}{\mathbb{K}}
%\newcommand{\D}{\mathbb{D}}
\renewcommand{\H}{\mathbb{H}}
\newcommand{\Cscr}{\mathscr{C}}
\newcommand{\Fscr}{\mathscr{F}}
\newcommand{\Kscr}{\mathscr{K}}
\newcommand{\Gscr}{\mathscr{G}}
\renewcommand{\Im}[1]{\text{Im}\left(#1\right)}
\renewcommand{\Re}[1]{\text{Re}\left(#1\right)}
\renewcommand{\^}[1]{\hat{#1}}
\newcommand{\dirv}[1]{\^{\vec #1}}
%\renewcommand{\exists}{\exists \phantom{-}}
\renewcommand{\t}{\left( t \right)}
\newcommand{\x}{\left( x \right)}
\newcommand{\y}{\left( y \right)}
\newcommand{\z}{\left( z \right)}
\newcommand{\partiell}[1]{\frac{\partial}{\partial #1} \ }
\newcommand{\partiellm}[2]{\frac{\partial^{#2}}{\partial {#1}^{#2}} \ }
\newcommand{\ins}{\subset}
\newcommand{\glatt}{$C^{\infty}$}
\newcommand{\hol}{\ \text{holomorph}}
\newcommand{\id}{\mathds{1}}
\newcommand{\Id}{\mathcal{I}}
\renewcommand{\dag}[1]{#1^{\dagger}}
\newcommand{\E}{\mathbb{E}}
\newcommand{\ot}{\otimes}
\renewcommand{\c}{\text{const.}}

\newcommand{\lb}{\nonumber \\}

%------- physics notation
\newcommand{\rGP}{\rho^{GP}_{N,a}}
\newcommand{\ket}[1]{|#1 \rangle}
\newcommand{\bra}[1]{\langle #1|}
\newcommand{\braket}[2]{\langle#1|#2\rangle}
\newcommand{\bk}[2]{\langle#1|#2\rangle}
\newcommand{\ketbra}[2]{|#1 \rangle\langle #2|}
\newcommand{\kb}[2]{|#1 \rangle\langle #2|}
\newcommand{\proj}[1]{|#1\rangle\langle #1|}
\newcommand{\dint}[1]{\mathrm{d}#1 \,}
\renewcommand{\bell}{\Phi^+}
\newcommand{\HiS}{\mathcal{H}}
\newcommand{\D}[1]{\mathcal{D}(#1)}
\renewcommand{\S}{\mathcal{S}}
\renewcommand{\P}{\mathcal{P}}
\newcommand{\SW}{\mathcal{SW}}
\newcommand{\PPT}{\Gamma}

% ------ Environments with Newtheorem

\theoremstyle{plain}
\newtheorem{thm}{Theorem}[section]
\newtheorem{conj}[thm]{Conjecture}
\newtheorem{prop}[thm]{Proposition}
\newtheorem{task}{Task}
%\newtheorem{lemma}[thm]{Lemma}
\newtheorem*{issue}{Issue}

\theoremstyle{definition}
%\newtheorem{definition}{Definition}

\theoremstyle{remark}
\newtheorem*{rem}{Remark}

% ------ New Operators

\DeclareMathOperator{\tr}{Tr}
\DeclareMathOperator{\End}{End}
\DeclareMathOperator{\Hom}{Hom}
\DeclareMathOperator{\Dim}{dim}
\DeclareMathOperator{\pr}{Pr}
\DeclareMathOperator{\diag}{diag}
\DeclareMathOperator{\conv}{Conv}
\DeclareMathOperator{\rank}{rank}
\DeclareMathOperator{\swap}{swap}
\DeclareMathOperator{\herm}{Herm}
\DeclareMathOperator{\cptpm}{CPTPM}
\DeclareMathOperator{\dist}{dist}
\DeclareMathOperator{\StateToVec}{StateToVec}
\DeclareMathOperator{\VecSwap}{VecSwap}
\DeclareMathOperator{\VecTr}{VecTr}
\DeclareMathOperator{\VecPT}{VecPT}
\DeclareMathOperator{\symm}{Symm}
\DeclareMathOperator{\Ran}{Ran}

% ------ griechische Buchstaben
\newcommand{\g}{\gamma}
\newcommand{\G}{\Gamma}
\renewcommand{\o}{\omega}
\renewcommand{\O}{\Omega}
\newcommand{\zt}{\zeta}
\renewcommand{\d}{\delta}
\newcommand{\De}{\Delta}
\newcommand{\p}{\phi}
\renewcommand{\b}{\beta}
\renewcommand{\a}{\alpha}
\renewcommand{\th}{\theta}
\newcommand{\s}{\sigma}
\newcommand{\e}{\epsilon}
\newcommand{\La}{\Lambda}

%\renewcommand\theparagraph{\textbf{paragraph}}
\newcommand{\norm}[1]{\left\|#1\right\|}
\newcommand{\abs}[1]{\left|#1\right|}
\newcommand{\oneo}[1]{\frac{1}{#1}}
\newcommand{\convS}{\stackrel{\S(\R^n)}{\longrightarrow}}

%%%%%%
% MISC
%%%%%%

\newcommand{\rev}[1]{\textbf{\textcolor{red}{!!! #1 !!!}}}


\usepackage[ngerman]{babel}
\usepackage{amsmath}
\usepackage{amssymb}
\usepackage[T1]{fontenc}
\usepackage{verbatim}
\usepackage{tikz}
\usetikzlibrary{arrows,shapes}
\usetikzlibrary{arrows,shapes,petri,topaths,through}
\tikzset{
	q_sys/.style={
		% shape
		rectangle, minimum size=6mm, rounded corners=3mm,
		% borders,filling and font
		very thick, draw = red!50!black!50!,top color=white,bottom color= red!50!black!20, font = \sffamily
		},
	op/.style={
		% shape
		rectangle, minimum size=6mm, rounded corners=3mm,
		% borders,filling and font
		very thick, draw = green!50!black!50!,top color=white,bottom color= green!20!black!10, font = \sffamily
		}
	}

\newtheorem*{bem}{Example}
\renewcommand{\vec}[1]{{\bf#1}}
%%%%%%%%%%%%%%%%%%%%%%%%%%%%%%%%%%%%%%%%%%%%%%%%%%%%%%%%%%%%

\title{Bound Entanglement}
%\subtitle{}
\author{Arne Hansen} %
\institute{ETH Zurich}
\date{02/2014}

%%%%%%%%%%%%%%%%%%%%%%%%%%%%%%%%%%%%%%%%%%%%%%%%%%%%%%%%%%%

\begin{document}
	
\tikzstyle{every picture}+=[remember picture]
\everymath{\displaystyle}

\begin{frame}
        \titlepage        
\end{frame}

%\begin{frame}
%	\tableofcontents
%\end{frame}

\section*{Intro}

\begin{frame}
\frametitle{Entanglement}
%\framesubtitle{``intriguing'', ``astonishing'', ``one of the most striking features of QM'',\ldots}
\begin{columns}
\column{1.5in}
\begin{figure}
\centering
\begin{tikzpicture}[scale=0.5]
	\coordinate (rho) at (4.2,2);
	%\coordinate (witness) at (5,-3);
	\coordinate (sep) at (-.5,0);
	\coordinate (dens) at (.5,4);
	\coordinate (ppt) at (0,2);	
	\filldraw[very thick,color=blue!70!black!90, fill=blue!50!black!50!] plot[smooth cycle,tension=0.6] coordinates{(-1.5,-1) (1.0,-2.5) (4,-2) (5.5,1) (5,3) (2,5) (-1.5,3)};
	% \filldraw[very thick,color=orange!80!black!90, fill=orange!50!black!50!] plot[smooth cycle,tension=0.6] coordinates{(-1.5,-1) (1.0,-2) (3.5,-0.5) (2,3) (-1.2,2)};
	\filldraw[very thick,color=red!80!black!90, fill=red!50!black!50!] plot[smooth cycle,tension=0.7] coordinates{(-1.5,-1) (1.0,-1.5) (0.8,1.2) (-1.2,1)};
	\filldraw [black] (rho) circle [radius=.05]; 
	\node [label=right:$\S$] (label) at (sep) {};
	% \node [label=right:$\PPT$] (label) at (ppt) {};
	\node [label=right:$\rho_{AB}$] (label) at (rho) {};
	\node [label=right:$\mathcal{D}$] (label) at (dens) {};
\end{tikzpicture}
%\caption{The set of separable states $\S$ is contained in the set of PPT states $\PPT$. Any NPT state $\rho$ is therefore entangled.}
\end{figure}
\column{1.5in}
\ldots peculiar, non-classical correlation arising from the tensor product structure in combined quantum systems
\end{columns}
% \visible<2>{
% \begin{block}{Bound Entanglement}
% 	non-distillable entanglement, associated with entangled PPT states
% 	\\ \vspace{5pt}\hfill \scriptsize (neglect possible NPT BE states)
% \end{block}}

\end{frame}

\begin{frame}
\frametitle{Bound Entanglement}
\begin{block}{Bound Entanglement}
	non-distillable entanglement %, associated with entangled PPT states
%	\\ \vspace{5pt}\hfill \scriptsize (neglect possible NPT BE states)
\end{block}
\vspace{10pt}
\begin{description}
\visible<2->{\item[Existence] Explicite example of a bound entangled state}
\visible<3->{\item[Use] Why should we be concerned about bound entanglement?}
\visible<4->{\item[Detection] How do we decide for a given state whether it is \visible<4>{bound} entangled?}
\end{description}
\end{frame}


\section[Existence of BE]{Existence Bound Entanglement}

\subsection{The PPT Criterion}

\begin{frame}
\frametitle{The PPT Criterion}
\framesubtitle{(Peres Criterion)}
\begin{columns}
\column{1.5in}
\visible<3->{\begin{figure}
\centering
\begin{tikzpicture}[scale=0.5]
	\coordinate (rho) at (4.2,2);
	%\coordinate (witness) at (5,-3);
	\coordinate (sep) at (-.5,0);
	\coordinate (dens) at (.5,4);
	\coordinate (ppt) at (0,2);	
	\filldraw[very thick,color=blue!70!black!90, fill=blue!50!black!50!] plot[smooth cycle,tension=0.6] coordinates{(-1.5,-1) (1.0,-2.5) (4,-2) (5.5,1) (5,3) (2,5) (-1.5,3)};
	\visible<4->{ \filldraw[very thick,color=orange!80!black!90, fill=orange!50!black!50!] plot[smooth cycle,tension=0.6] coordinates{(-1.5,-1) (1.0,-2) (3.5,-0.5) (2,3) (-1.2,2)};}
	\filldraw[very thick,color=red!80!black!90, fill=red!50!black!50!] plot[smooth cycle,tension=0.7] coordinates{(-1.5,-1) (1.0,-1.5) (0.8,1.2) (-1.2,1)};
	\filldraw [black] (rho) circle [radius=.05]; 
	\node [label=right:$\S$] (label) at (sep) {};
	\visible<4->{\node [label=right:$\PPT$] (label) at (ppt) {};}
	\node [label=right:$\rho_{AB}$] (label) at (rho) {};
	\node [label=right:$\mathcal{D}$] (label) at (dens) {};
\end{tikzpicture}
%\caption{The set of separable states $\S$ is contained in the set of PPT states $\PPT$. Any NPT state $\rho$ is therefore entangled.}
\end{figure}}
\column{2.8in}

% The partial transpose of a multi-partite system, i.e. the operator transposing a selection of the subsystems, serves as indicator of entanglement. The following definition refers to the bipartite case but generalizes easily to more parties.

\only<1-4>{
\begin{definition} 
	The partial transpose of a density matrix $\rho\in \D{\HiS_A\ot\HiS_B}$ is defined as
	\begin{equation}
		\rho^{\G} := (\Id_A \ot T_B) \rho
	\end{equation}
\end{definition}

\begin{block}{Matrix elements in the tensor basis}
  In the product basis
  \begin{IEEEeqnarray}{RL}
    \left(\rho^{\G}\right)_{ikjl} &= \bra{i \otimes k} \ \rho^{\G} \ \ket{j \otimes l} \\
    &= \bra{i \otimes l} \ \rho \ \ket{j \otimes k} = \rho_{iljk}
  \end{IEEEeqnarray}
\end{block}
}

\only<5-6>{\visible<5-6>{\begin{alertblock}{Entanglement criterion}
  Any non-PPT(NPT) density matrix is entangled.
\end{alertblock}}

\visible<6>{\begin{block}{Group of Entanglement Tests}
  Any Positive but not Completely Positive Map ('channel’) serves as a similar entanglement test.
\end{block}}}

\end{columns}
\end{frame}
\subsection{Introducing Bound Entanglement}
\begin{frame}

\frametitle{Example of a bound entangled state}


\end{frame}


\section[Applications]{Applications of BE}

\begin{frame}
\frametitle{Quantum Key Distribution}
\framesubtitle{with bound entangled resources}
\end{frame}

\section[Detection]{Testing bound entanglement}

\begin{frame}
\frametitle{Deciding entanglement for PPT}
\end{frame}


\section[Quant Repeaters]{Notions of Quantum Repeaters}

\begin{frame}
	\tableofcontents[currentsection,subsectionstyle=show/show/hide,subsubsectionstyle=show/show/hide]
\end{frame}

\begin{frame}
\frametitle{Endeavour: Extending entanglement}

\only<1-2>{
\begin{figure}
\centering
\begin{tikzpicture}[]
	% nodes
	\node (A) at (0,0) [q_sys]	{A};
	\node (C) at (7,0) [q_sys]	{C};
	% paths
	%\y=1;
	%\draw[very thick,color=orange!80!black!90] plot[smooth cycle,tension=0.95] coordinates{(2.5,1) (3.5,1.5) (4.5,1) (3.5,0.5)};
	\alert<2>{\draw[thick] (A.east) -- (C.west);
	\visible<2>{\node (noise) at (3.5,0.3) {noise};}}
\end{tikzpicture}
%\caption{Principles behind quantum repeaters}
\end{figure}
}

\only<3->{
\begin{figure}
\centering
\begin{tikzpicture}[]
	% nodes
	\foreach \x in {0,1}	{
		\node (A_\x) at (0,\x) [q_sys]	{A};
		\node (B_\x) at (3,\x) [q_sys]	{B} \ifnum \x > 0 
														edge[thick] (A_\x) \fi;
		\node (B_prime) at (4,\x) [q_sys]	{B'};
		\node (C_\x) at (7,\x) [q_sys]	{C} \ifnum \x > 0 
														edge[thick] (B_prime)\fi;
	}
	% paths
	%\y=1;
	%\draw[very thick,color=orange!80!black!90] plot[smooth cycle,tension=0.95] coordinates{(2.5,1) (3.5,1.5) (4.5,1) (3.5,0.5)};
	\draw[thick] plot[smooth, tension=0.95] coordinates{(A_0.east) (3.5,-0.7) (C_0.west)};
\end{tikzpicture}
%\caption{Principles behind quantum repeaters}
\end{figure}
}
\end{frame}

\subsection{Swap Operation on States}
%\subsection{The Grand Canonical Ensemble}



\note{different perceptions of quantum repeaters --> theoretical approaches}

\begin{frame}
\frametitle{Swap Operator on States}

\begin{figure}
\centering
\begin{tikzpicture}[]
	% nodes
	\only<-2>{\node (labelA) at (1.5,0.3) {$\rho_1$};
	\node (labelB) at (5.5,0.3) {$\rho_2$};}
	\node (A) at (0,0) [q_sys]	{A};
	\node (B) at (3,0) [q_sys]	{B};
	\node (B_prime) at (4,0) [q_sys] {B'};
	\node (C) at (7,0) [q_sys]	{C};
	\visible<1-2>{\draw[thick,color=black] (A.east) -- (B.west);
		\draw[thick,color=black] (B_prime.east) -- (C.west); }
	% paths
	\visible<2>{\draw[very thick,color=orange!80!black!90] plot[smooth cycle,tension=0.95] coordinates{(2.5,0) (3.5,.5) (4.5,0) (3.5,-0.5)};}
	\visible<3->{\draw[thick] plot[smooth, tension=0.95] coordinates{(A.east) (3.5,-0.7) (C.west)};}
\end{tikzpicture}
%\caption{Principles behind quantum repeaters}
\end{figure}
\visible<4->{
\begin{block}{Swap operation}
\[
	\swap(\rho_A,\rho_B) = \oneo{N} \tr_{BB'}\left[ \ (\proj{\bell}_{BB'} \ot \id_{AC} ) \ (\rho_1 \ot \rho_2) \ \right]
\]
\end{block}}

\end{frame}

\subsection{Composition of Channels}

\begin{frame}
\frametitle{Composing Channels}
\framesubtitle{corresponds to swapping CJ associated states}

\begin{figure}
\centering
\begin{tikzpicture}
\node (A1) at (-.2,0) [q_sys] {A};
\node (A2) at (2,0) [q_sys] {A'};
\visible<2->{\node (B) at (5,0) [q_sys] {B};}
\visible<4->{\node (C) at (8,0) [q_sys] {C};}
\draw[thick, dashed]  (A1) -- node[above] {$\proj{\bell}$} (A2) {};
\visible<3->{\draw[thick,->]  (A2) -- node[above] {$\tau^{-1}(\rho_1)$} (B) {};}
\visible<5->{\draw[thick,->]  (B) -- node[above] {$\tau^{-1}(\rho_2)$} (C) {};}
\end{tikzpicture}
%\caption{Alice prepares a Bell state system on $A\ot A'$ and sends one part to Bob who forwards this to Charlie.}
\end{figure}
\visible<6>{
\begin{block}{Swap in terms of channels}
	\[
	\tau^{-1}(\swap(\rho_1,\rho_2)) = \tau^{-1}(\rho_2)\circ\tau^{-1}(\rho_1)
	\]
\end{block}
(flipping systems or equivalently transposing channels if necessary, up to normalization)
}

\end{frame}

\subsection{Hybrid Approach}

\begin{frame}
\frametitle{Hybrid approach}

\begin{figure}
\centering
\begin{tikzpicture}
\node (A) at (0,0) [q_sys] {A};
\node (B) at (2,0) [q_sys] {B};
\visible<2->{\node (C) at (5,0) [q_sys] {C};}
\draw[thick, dashed]  (A) -- node[above] {$\rho_1$} (B) {};
\visible<3->{\draw[thick,->]  (B) -- node[above] {$\tau^{-1}(\rho_2)$} (C) {};}
\end{tikzpicture}
%\caption{In the hybrid approach Bob sends his part of an entangled system he shares with Alice to Charlie.}
\end{figure}
\visible<4>{
\begin{block}{Hybrid formulation}
	\[\swap(\rho_1,\rho_2) = \left[\Id_A\ot\tau^{-1}(\rho_2)\right](\rho_1) = \left[\Id_C\ot\tau^{-1}(\rho_1)^T\right](\rho_2)\]
\end{block}
(flipping systems or equivalently transposing channels if necessary, up to normalization)}
\end{frame}

\section{Question}

%\begin{frame}
%	\tableofcontents[currentsection]
%\end{frame}


\begin{frame}
\frametitle{Swapped Bound Entanglement}
\framesubtitle{Entanglement of swapped states from BE resources?}

\begin{block}{States: Image of the Swap operator}
	\[ \swap: \text{BE states} \times \text{BE states} \to \alert{\S} \]
	\centerline{OR}
	\[ \swap(\text{BE states},\text{BE states}) \cap  \mathcal{D}\setminus\S\alert{\neq \emptyset}\]
\end{block}

\visible<2>{

\begin{block}{Composed Channels}
	If $\La_1: \D{A} \to \D{B}$ and $\La_2: \D{B} \to \D{C}$ are PPT channels\footnote{Defined by their image: $\La_{\G}(\mathcal{D})\subset \PPT$}, is $\alert{\La_2\circ\La_1}:\D{A} \to \D{C}$ \alert{separable}?
\end{block}
}
\end{frame}

\subsection{Motivation}

\begin{frame}
\frametitle{Why should we bother?}

\begin{description}[<+->]
\item[QKD] distribute secure keys over larger distances \cite{Sec_key_b_ent}
\item[Channels] range of PPT channels \\ (image of PPT states under PPT channels)
\item[States] preservation of bound entanglement under swapping
\end{description}
\end{frame}

\section{Answers}

\begin{frame}
\frametitle{Answers}
\pause
\centerline{\large Well, \ldots	}

%\par \par \par \pause
%\flushright{\small (\ldots but is slightly more to say.)}

\end{frame}



\end{document}
